\subsection{Installing Papaya Backend}\label{subsec:installing-papaya-backend}

Papaya Backend provides two ways of launching: using WAR deployment or
a bootable JAR\@.
Normally the bootable JAR is easier to use and set up.
However, the WAR deployment allows more advanced setups.

\subsubsection{Using bootable JAR}

In order to launch Papaya Backend using bootable JAR, it's enough to launch
the JAR file:

\begin{lstlisting}
    java -jar papaya-web.jar
\end{lstlisting}

See~\ref{subsec:configuring-papaya-backend} for configuration details.


\subsubsection{Using deployable WAR}

In order to launch Papaya Backend using deployable WAR, an application server,
such as Tomcat, is needed.

\begin{enumerate}
    \item Move \texttt{papaya-web.war} to
    \texttt{\$CATALINA\_BASE/webapps/api.war},

    \item Run Tomcat using \texttt{\$CATALINA\_BASE/bin/startup.sh}.
\end{enumerate}

After the steps above, Papaya Backend should be deployed under \texttt{/api}
on the Tomcat server.

Further information about Tomcat configuration is available at:
\url{http://tomcat.apache.org/tomcat-9.0-doc/index.html}.

It is possible and advisable to deploy both Papaya Frontend and Papaya Backend
on the same server, as it greatly simplifies further configuration.

See~\ref{subsec:configuring-papaya-backend} for configuration details.


\subsection{Configuring Papaya Backend}\label{subsec:configuring-papaya-backend}

Generally it is possible to configure Papaya Backend using either environment
variables or replacing a configuration file.

\subsubsection*{Environment variables}

Before running the Papaya Backend the following environment variables may be
configured:

\begin{itemize}
    \setlength\itemsep{0em}
    \item \texttt{PAPAYA\_DB\_URL} (example: \texttt{jdbc:postgresql://localhost:5432/papaya\_db}) \\
    Database URL \\
    \item \texttt{PAPAYA\_DB\_USERNAME} (example: \texttt{postgres}) \\
    Database user name
    \item \texttt{PAPAYA\_DB\_PASSWORD} (example: \texttt{postgres}) \\
    Database user password
    \item \texttt{PAPAYA\_EMAIL\_HOST} (example: \texttt{smtp.example.com}) \\
    SMTP server address user for sending emails
    \item \texttt{PAPAYA\_EMAIL\_ADDRESS} (example: \texttt{example@example.com}) \\
    Email address used as a sender in sent emails
    \item \texttt{PAPAYA\_EMAIL\_PASSWORD} (example: \texttt{example}) \\
    \item \texttt{PAPAYA\_EMAIL\_PORT} (example: \texttt{587}) \\
    Port of the SMTP server
    \item \texttt{PAPAYA\_KEYCLOAK\_REALM} (example: \texttt{papaya}) \\
    Keycloak realm used for authentication
    \item \texttt{PAPAYA\_KEYCLOAK\_ADDR} (example: \texttt{http://localhost:9990/auth}) \\
    Keycloak server address
    \item \texttt{PAPAYA\_KEYCLOAK\_CLIENT} (example: \texttt{papaya-web}) \\
    Keycloak client ID used by Papaya
    \item \texttt{PAPAYA\_KEYCLOAK\_SSL\_REQUIRED} (example: \texttt{none}) \\
    SSL configuration for Keycloak
    \item \texttt{PAPAYA\_KEYCLOAK\_CLIENT\_SECRET} (example: \texttt{d3dERmmALg62evkS}) \\
    If client requires a secret, the value configured here will be used therein
    \item \texttt{PAPAYA\_KEYCLOAK\_CLIENT\_USERNAME} (example: \texttt{papaya-cli}) \\
    Name of the service user used by Papaya Backend to access Keycloak data
    \item \texttt{PAPAYA\_KEYCLOAK\_CLIENT\_PASSWORD} (example: \texttt{papaya}) \\
    Password of the service user
    \item \texttt{JIRA\_AUTH\_PRIVATE\_KEY}  \\
    \item \texttt{JIRA\_AUTH\_PUBLIC\_KEY}  \\
\end{itemize}

\subsubsection*{Configuration file}

Inside the Papaya Backed deployment, it is possible to override the file
\texttt{WEB-INF/classes/application-prod.properties}
which contains all the above configuration and set the values manually.
