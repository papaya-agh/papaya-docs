\section{Obtaining Papaya}\label{sec:obtaining-papaya}

\subsection{Downloading pre-built artifacts}\label{subsec:downloading-pre-built-artifacts}

Pre-built artifacts are available under the ``GitHub Releases'' pages of the
Papaya project repositories:
\begin{itemize}
    \item Backend:
    \href{https://github.com/papaya-agh/papaya-backend/releases}{papaya-agh/papaya-backend}

    \item Frontend:
    \href{https://github.com/papaya-agh/papaya-frontend/releases}{papaya-agh/papaya-frontend}
\end{itemize}

\emph{Note: the downloaded Frontend will be pre-configured to search for Keycloak under \texttt{/auth}.}

\subsection{Building from source}\label{subsec:building-from-source}

\subsubsection*{Papaya Backend}

\begin{enumerate}
    \item Clone the GitHub repository:
    \begin{lstlisting}[gobble=8]
        git clone https://github.com/papaya-agh/papaya-backend
        cd papaya-backend
    \end{lstlisting}

    \item Build the project using Gradle wrapper:
    \begin{lstlisting}[gobble=8]
        ./gradlew build -x test
    \end{lstlisting}

    \item (optional) Build the bootable JAR file:
    \begin{lstlisting}[gobble=8]
        ./gradlew bootJar -x test
    \end{lstlisting}
\end{enumerate}

After those steps, the artifacts are available at
\texttt{./papaya-web/build/libs/}.

\subsubsection*{Papaya Frontend}

\begin{enumerate}
    \item Clone the GitHub repository:
    \begin{lstlisting}[gobble=8]
        git clone https://github.com/papaya-agh/papaya-frontend
    \end{lstlisting}

    \item Initialize submodules:
    \begin{lstlisting}[gobble=8]
        cd papaya-frontend
        git submodule init
    \end{lstlisting}

    \item Install Angular and other dependencies:
    \begin{lstlisting}[gobble=8]
        npm install -g @angular/cli
        npm install
    \end{lstlisting}

    \item Generate required code:
    \begin{lstlisting}[gobble=8]
        npm run generate.declarations
    \end{lstlisting}

    \item (optional) Set Keycloak URL, realm, and client ID in
    \texttt{src/environments/environment.prod.ts}

    \item Build and create the desired .war file:
    \begin{lstlisting}[gobble=8]
        ng build --prod
        npm run war
    \end{lstlisting}
\end{enumerate}

After those steps, the artifact is available at \texttt{./dist/}.
